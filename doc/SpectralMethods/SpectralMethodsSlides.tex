
\documentclass{beamer}
\usetheme{default}


\title{Spectral Methods in MATLAB}


\subtitle{On infinite domains}
\author{Pablo Brubeck\inst{1}}
\institute[ITESM]
{
  \inst{1}%
  Department of Physics\\
  Tecnologico de Monterrey
}

\date{\today}
\subject{Numerical Analysis}

\AtBeginSection[]
{
  \begin{frame}<beamer>{Outline}
    \tableofcontents[currentsection]
  \end{frame}
}

% Let's get started
\begin{document}

\begin{frame}
  \titlepage
\end{frame}

\begin{frame}{Outline}
  \tableofcontents
  % You might wish to add the option [pausesections]
\end{frame}

% Section and subsections will appear in the presentation overview
% and table of contents.
\section{Function representation}

\subsection{Cardinal Functions}

\begin{frame}{A function may be expressed as\\ a linear combination of cardinal functions.}{}
\begin{equation*}
f(x) = \sum_{j=1}^n f(x_j) C_j(x)
\end{equation*}

\begin{equation*}
C_j(x_i) = \delta_{ij}=\begin{cases}1&i=j\\0&i\neq j\end{cases}
\end{equation*}
\end{frame}

\begin{frame}{The nodes correspond to the zeros\\ of some special polynomial.}{}
\begin{equation*}
P_n(x_i) = 0
\end{equation*}

\begin{equation*}
C_j(x) = \frac{P_n(x)}{(x-x_j)P_n'(x_j)}
\end{equation*}
\end{frame}


\subsection{Differentiation Matrices}

% You can reveal the parts of a slide one at a time
% with the \pause command:
\begin{frame}{The derivate of a function is the combination of\\ the derivates of the cardinal functions.}
\begin{equation*}
\frac{d}{dx}f(x)|_{x_i} = \sum_{j=1}^n f(x_j) C'_j(x_i)
\end{equation*}
\end{frame}

\section{Orthogonal Polynomials}

\begin{frame}{Ortogonal polynomials form great bases.}
\begin{equation*}
    \int_R P_m(x)P_n(x) w(x)dx = c_m\delta_{mn}
\end{equation*}

\end{frame}

\subsection{Hermite}

\begin{frame}{Hermite polynomials are ideal\\ for infinite domains.}
\begin{equation*}
    H_n(x)
\end{equation*}
\begin{equation*}
    x \in (-\infty, \infty)
\end{equation*}
\begin{equation*}
    w(x) = e^{-x^2}
\end{equation*}

% Figure of Hermite nodes

\end{frame}

\subsection{Laguerre}

\begin{frame}{Laguerre polynomials are ideal\\ for the radial coordinate.}
\begin{equation*}
    L_n^{1}(x)
\end{equation*}
\begin{equation*}
    r \in [0, \infty)
\end{equation*}
\begin{equation*}
    w(r) = re^{-r}
\end{equation*}

% Figure of Laguerre nodes

\end{frame}


\section{Ordinary Differential Equations}
\subsection{Boundary Value Problems}
\subsection{Eigenvalue problems}

\section{Partial Differential Equations}
\subsection{Elliptic Differential Equations}
\subsection{Parabolic Differential Equations}

% Placing a * after \section means it will not show in the
% outline or table of contents.
\section*{Summary}

\begin{frame}{Summary}
  \begin{itemize}
  \item
    The \alert{first main message} of your talk in one or two lines.
  \item
    The \alert{second main message} of your talk in one or two lines.
  \item
    Perhaps a \alert{third message}, but not more than that.
  \end{itemize}
  
  \begin{itemize}
  \item
    Outlook
    \begin{itemize}
    \item
      Something you haven't solved.
    \item
      Something else you haven't solved.
    \end{itemize}
  \end{itemize}
\end{frame}



% All of the following is optional and typically not needed. 
\appendix
\section<presentation>*{\appendixname}
\subsection<presentation>*{For Further Reading}

\begin{frame}[allowframebreaks]
  \frametitle<presentation>{For Further Reading}
    
  \begin{thebibliography}{10}
    
  \beamertemplatebookbibitems
  % Start with overview books.

  \bibitem{Author1990}
    A.~Author.
    \newblock {\em Handbook of Everything}.
    \newblock Some Press, 1990.
 
    
  \beamertemplatearticlebibitems
  % Followed by interesting articles. Keep the list short. 

  \bibitem{Someone2000}
    S.~Someone.
    \newblock On this and that.
    \newblock {\em Journal of This and That}, 2(1):50--100,
    2000.
  \end{thebibliography}
\end{frame}

\end{document}


