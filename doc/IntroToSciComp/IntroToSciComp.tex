\documentclass[xcolor={dvipsnames}]{beamer}
\usepackage[utf8]{inputenc}
\usepackage{amsmath, amssymb, upgreek}
\usepackage{graphicx, float, subcaption}
\usepackage{hyperref}

\usetheme{default}


\title{Introduction to Scientific Computing}


\author{Pablo Brubeck\inst{1}}
\institute[ITESM]
{
  \inst{1}%
  Department of Physics\\
  Tecnologico de Monterrey
}

\date{\today}
\subject{Numerical Analysis}

\AtBeginSection[]
{
  \begin{frame}<beamer>{Outline}
    \tableofcontents[currentsection]
  \end{frame}
}

\begin{document}

\begin{frame}
  \titlepage
\end{frame}

\begin{frame}{Outline}
  \tableofcontents
  % You might wish to add the option [pausesections]
\end{frame}


\section{The Big Picture}

\begin{frame}{The 4 steps of Scientific Computing}{}
\begin{enumerate}
\item Problem
\item Equation
\item Algorithm
\item Visualization
\end{enumerate}
\end{frame}


\begin{frame}{Scientific Computing problems\\ may be classified into:}{}
\begin{itemize}
\item Root finding (Solve a system of equations)
\item Optimization (Maximize or minimize a cost function)
\item Modeling of data (Curve fitting, interpolation, extrapolation)
\item Numerical evaluation (Functions, series, derivatives, integrals)
\item Differential equations (Boundary Value Problems, Initial Value Problems)
\end{itemize}
\end{frame}

\begin{frame}{Approaches of Scientific Computing}{}
\begin{itemize}
	\item Analytical
	\item Numerical
	\item Algorithmic
	\item Heuristic
\end{itemize}
\end{frame}


\section{Tools and Environments}

\subsection{Methodology}

\begin{frame}{Solve the problem first,\\ then write code.}{}

\end{frame}

\subsection{Which language should I use?}

\begin{frame}{High-level programming provides\\ the most straight-foward solutions}{}
A high-level approach relies on abstraction and a programming language rich in functions.

\bigskip
\begin{itemize}
\item Python (Fuctional)
\item MATLAB (Numerical)
\item COMsolve (Finite Element Method)
\item Mathematica (Functional)
\end{itemize}
\end{frame}

\begin{frame}{Low-level programming provides\\ the most efficient solutions}{}
A low-level approach ussually requires more lines of code, but it is the way to achieve the most efficient solutions.

\bigskip
\begin{itemize}
\item Fortran (Old school, the latin of programming languages)
\item C++ (Universal)
\item Java (Multi-platform applications)
\end{itemize}

\bigskip
Explicit data types and memory management.
\end{frame}

\subsection{Best practices}
\begin{frame}{Meaningful function and variable names\\ may save your life.}{}

Naming conventions
Consistency


\end{frame}


\begin{frame}{Tips and tricks.}{}
\begin{itemize}
\item You can indent several lines of code at once.
\item Decrease indent shortcut (Shift-Tab)
\end{itemize}
\end{frame}


\subsection{Libraries}
\begin{frame}{Commonly, your problem has already been solved\\ or at least partially solved}{}

The internet was invented for scientific collaboration. Google is your friend!
 
\bigskip
Here are some usefull sites
\begin{itemize}
\item Stack Overflow
\item Stack Exchage (Super User, Mathematics, Physics)
\item Physics Forums
\item Online documentation
\end{itemize}

\end{frame}

\begin{frame}{Popular Scientific Computing libraries}{}
User firendly
\begin{itemize}
	\item NumPy, SciPy, matplotlib (MATLAB inside Python)
	\item Armadillo (MATLAB inside C++)
	\item Plotly (Online data visualization)
\end{itemize}

\bigskip
State of the art (God Tier)
\begin{itemize}
	\item Intel MKL (Math Kernel Library)
	\item GSL (GNU Scientific Library for C/C++)
	\item BLAS and LAPACK (Linear Algebra)
	\item FFTW (Fastest Fourier Transform in the West)
	\item CUDA, OpenCL (GPU programming)
	\item OpenGL, Vulkan, Direct3D (graphics)
\end{itemize}
\end{frame}


\begin{frame}{Specialized Scientific Computing libraries}{}

\begin{itemize}
\item Chebfun (Spectral Methods)
\item NFFT (Non-equispaced Fast Fourier Transform)
\item ARPACK (Sparse Eigensolver)
\item SCPACK (Schwartz-Christoffel Conformal mapping)
\item DistMesh (Simple mesh generator in MATLAB)

\end{itemize}
\end{frame}




\section{Bondary Value Problems}

\begin{frame}{A Boundary Value Problem (BVP) \\ is a differential equation}{}

Find $y(x)$ that satifies a differential equation and boundary conditions.
\end{frame}


\begin{frame}{Many natural-occuring BVPs are second order and linear}{}

Linear problems

\end{frame}


\end{document}


